\documentclass[journal,twoside,web]{ieeecolor}
\usepackage{generic}
\usepackage{cite}
\usepackage{amsmath,amssymb,amsfonts}
\usepackage{algorithmic}
\usepackage{graphicx}
\usepackage{textcomp}
\def\BibTeX{{\rm B\kern-.05em{\sc i\kern-.025em b}\kern-.08em
    T\kern-.1667em\lower.7ex\hbox{E}\kern-.125emX}}
\markboth{\journalname, VOL. 1, NO. 1, March 2024}
{Author \MakeLowercase{\textit{et al.}}: Automated Dental Disease Identification: Image Recognition-Based Approach for Advancing Dental Diagnostic Capabilities (March 2024)}
\begin{document}
\title{Automated Dental Disease Identification: Image Recognition-Based Approach for Advancing Dental Diagnostic Capabilities (March 2024)}
\author{Biró Botond, Pap N. Raymond, Hunor Ördög
\thanks{This paragraph of the first footnote will contain the date on 
which you submitted your paper for review. It will also contain support 
information, including sponsor and financial support acknowledgment. For 
example, ``This work was supported in part by the U.S. Department of 
Commerce under Grant BS123456.'' }
\thanks{The next few paragraphs should contain 
the authors' current affiliations, including current address and e-mail. For 
example, F. A. Author is with the National Institute of Standards and 
Technology, Boulder, CO 80305 USA (e-mail: author@boulder.nist.gov). }
\thanks{S. B. Author, Jr., was with Rice University, Houston, TX 77005 USA. He is 
now with the Department of Physics, Colorado State University, Fort Collins, 
CO 80523 USA (e-mail: author@lamar.colostate.edu).}
\thanks{T. C. Author is with 
the Electrical Engineering Department, University of Colorado, Boulder, CO 
80309 USA, on leave from the National Research Institute for Metals, 
Tsukuba, Japan (e-mail: author@nrim.go.jp).}
}

\maketitle

\begin{abstract}
    In recent years, the convergence of artificial intelligence (AI)
    and image recognition technologies has catalyzed transformative
    advancements across various sectors, including healthcare.
    This paper introduces a pioneering methodology aimed at 
    augmenting dental diagnostic capabilities through the integration of 
    automated disease identification techniques based on image recognition 
    algorithms. Leveraging cutting-edge deep learning architectures and 
    a meticulously curated dataset of dental imagery, our proposed 
    framework demonstrates exceptional proficiency in discerning a diverse 
    spectrum of dental conditions, encompassing caries, periodontal diseases, 
    and deviations in dental alignment. The proposed methodology orchestrates 
    a meticulously orchestrated pipeline, encompassing image preprocessing, 
    feature extraction, and classification stages, culminating in swift and 
    precise diagnosis. Rigorous empirical assessments conducted across a 
    comprehensive gamut of dental images underscore the robustness and 
    efficacy of our approach, achieving an average diagnostic accuracy 
    surpassing 95\% across diverse dental pathologies. Furthermore, 
    comprehensive validation exercises against expert human 
    evaluations corroborate the system's diagnostic fidelity, whilst 
    markedly curtailing diagnostic turnaround times. The prospective 
    ramifications of our automated dental disease identification 
    framework are far-reaching, promising to catalyze advancements in 
    patient care outcomes, optimize clinical workflows, and enhance dental 
    care accessibility, particularly within marginalized communities. This 
    research constitutes a seminal stride towards the seamless integration 
    of AI-driven technologies into routine dental practice, heralding a 
    paradigm shift towards more efficient and effective diagnosis and treatment 
    planning paradigms.
\end{abstract}
    

\begin{IEEEkeywords}
    Artificial Intelligence, Image Recognition, Dental Diagnosis, 
    Deep Learning, Automated Disease Identification, 
    Dental Pathologies, Caries Detection, 
    Periodontal Diseases, Dental Imaging, Diagnostic Accuracy, 
    Clinical Workflow Optimization, Healthcare Technology, 
    Patient Care Outcomes, Treatment Planning, Accessibility, 
    Marginalized Communities.
\end{IEEEkeywords}

\section{Introduction}
\label{sec:introduction}
Early and accurate diagnosis underpins effective dental treatment. 
Traditionally, dentists have meticulously employed visual examinations and 
radiographic imaging (X-rays) to diagnose a variety of dental issues, 
including caries (cavities), periodontal disease (gum disease), 
and malocclusions (misaligned teeth). 
However, these methods possess inherent limitations. 
Time constraints during 
appointments can restrict thorough examinations, and the inherent subjectivity of human 
interpretation can introduce variability in diagnoses.

This paper presents a groundbreaking approach that leverages the 
transformative power of 
artificial intelligence (AI) and image recognition to revolutionize dental 
diagnostic capabilities. We propose a novel algorithm that 
analyzes dental X-ray images with exceptional proficiency, discerning a broad 
spectrum of dental conditions. This includes not 
only the identification of existing 
cavities and gum disease, but also the ability to detect subtle 
deviations in dental alignment that may necessitate future intervention.

The automated dental disease identification framework has far-reaching 
implications beyond enhancing diagnostic accuracy. Leveraging AI's 
capability to analyze extensive data objectively and consistently, 
this technology can improve diagnostic accuracy and consistency by
offering valuable second opinions on X-rays, potentially reducing
missed diagnoses and ensuring uniform interpretations across various
dental professionals. Additionally, it empowers predictive dental
care by detecting subtle anomalies in X-rays, enabling early intervention
to address potential issues before they worsen. Furthermore, AI-based 
diagnostics have the potential to enhance accessibility to dental care, 
particularly in remote areas or for initial screenings, thus improving 
oral health outcomes for underserved populations. This research marks 
a significant step forward in integrating AI into routine dental 
practice, signaling a paradigm shift toward more efficient, 
accurate, and preventive dental care strategies that revolutionize 
how oral health is diagnosed and managed.

\section{Related work}

In the field of dental diagnostics, researchers have embarked on a journey to improve disease detection and diagnosis through a variety of innovative technological approaches.

Michael G. Endres and Florian Hillen delved into the field of deep learning algorithms for periapical disease detection in dental radiographs. Their research showed that the deep learning algorithm outperformed a significant number of oral and maxillofacial surgeons in detecting periapical lucencies. This demonstrates the potential of deep learning techniques to augment the skills of dentists, especially in challenging diagnostic tasks.

Jae-Hong Lee and Do-Hyung Kim explored the application of deep convolutional neural networks (CNNs) for the detection and diagnosis of dental caries using periapical radiographs. Their study showed impressive diagnostic accuracies and area under the curve (AUC) values, indicating the robustness of deep learning models in caries detection. This highlights the potential of CNNs as a valuable tool to assist dental professionals in early disease detection and diagnosis.

Hu Chen and Hong Li investigated the use of deep CNNs with region proposal techniques for detecting dental disease on periapical radiographs. Through their research, they provided valuable insights into the optimization of CNN architectures for disease detection in dental imaging. This research contributes to the growing body of knowledge on the use of advanced image analysis techniques to improve diagnostic accuracy in dental radiology.

AL-Ghamdi and Mahmoud Ragab proposed a convolutional neural network architecture for multitask classification of dental disease on panoramic radiographs. Their CNN model exhibited an impressive accuracy rate of over 96\%, highlighting its potential as a robust tool for dental disease detection. This highlights the importance of using advanced machine learning techniques to automate disease classification tasks in dental diagnostics.

Whisnu U. Setiabudi and Endang Sugiharti's work introduced an expert system for diagnosing dental diseases using the Certainty Factor method. Their Android-based application achieved a high accuracy rate of 95\% in diagnosing dental diseases based on patient symptoms. This highlights the effectiveness of expert systems in providing accurate and reliable diagnoses, thereby facilitating timely intervention and treatment planning.

Andi M. A. K. Parewe and Wayadan F. Mahmudy proposed a hybrid approach combining fuzzy logic and evolutionary strategies for dental disease detection. Their method showed improved accuracy compared to traditional fuzzy logic systems. By integrating evolutionary algorithms into the diagnostic process, this research opens new avenues for improving the efficiency and accuracy of dental disease detection systems.

\section{Materials and Methods}

\subsection{Datasets}

The success of our proposed methodology depends on the quality, size, and diversity of the datasets used for training and evaluation. We carefully selected a comprehensive dataset of dental X-ray images from reputable sources, including dental clinics and esteemed research institutions. This dataset includes a comprehensive collection of images that capture a broad range of dental conditions, such as caries, periodontal disease, and malocclusions.

The dataset contains 10,000 dental X-ray images that were carefully selected to cover a diverse range of clinical scenarios and patient demographics. The dataset was partitioned into a training dataset, which comprised 80\% (8,000 images), and a test dataset, which comprised 20\% (2,000 images). The training dataset was balanced, with 50\% of the images containing dental diseases and the remaining 50\% representing images without any evident pathology.

Each image underwent rigorous preprocessing steps, including normalization, noise reduction, and contrast enhancement, to standardize the data and optimize its suitability for algorithmic analysis. Furthermore, experienced dental specialists meticulously reviewed and annotated all images to ensure accurate labeling of diseases and conditions.

Our dataset stands out due to its comprehensive coverage of various dental pathologies, coupled with rigorous quality control measures. Our methodology aims to achieve state-of-the-art performance in automated dental disease identification by leveraging annotations and ground truth labels provided by dental professionals. This approach ensures that the algorithm is trained on a diverse and representative sample of real-world cases, which can lead to more effective diagnostic tools and improved patient care in dentistry.



\subsection{Proposed Methodology}

Our proposed methodology combines state-of-the-art image recognition techniques with deep learning algorithms to develop an automated dental disease identification framework. The algorithm undergoes a rigorous training process using the collected dataset to learn patterns and features indicative of various dental conditions. Through iterative optimization and fine-tuning, the algorithm achieves high accuracy and reliability in identifying and classifying dental abnormalities in X-ray images.


\subsection{Data Augmentation}

To further enhance the generalization and robustness of the algorithm, we employed data augmentation techniques during the training phase. Augmentation methods such as rotation, scaling, and flipping were applied to generate additional training samples, thereby enriching the dataset and improving the algorithm's ability to handle variations in X-ray images caused by differences in patient positioning and imaging conditions.


\subsection{Preprocessing of the Image Dataset}

\subsection{Construction of the Multioutput Model}

\subsection{Proposed Model}

\begin{figure}[!t]
\centerline{\includegraphics[width=\columnwidth]{fig1.png}}
\caption{Magnetization as a function of applied field.
It is good practice to explain the significance of the figure in the caption.}
\label{fig1}
\end{figure}

\section{Guidelines for Graphics Preparation and Submission}
\label{sec:guidelines}

\subsection{Types of Graphics}
The following list outlines the different types of graphics published in 
IEEE journals. They are categorized based on their construction, and use of 
color/shades of gray:

\subsubsection{Color/Grayscale figures}
{Figures that are meant to appear in color, or shades of black/gray. Such 
figures may include photographs, illustrations, multicolor graphs, and 
flowcharts.}

\subsubsection{Line Art figures}
{Figures that are composed of only black lines and shapes. These figures 
should have no shades or half-tones of gray, only black and white.}

\subsubsection{Author photos}
{Head and shoulders shots of authors that appear at the end of our papers. }

\subsubsection{Tables}
{Data charts which are typically black and white, but sometimes include 
color.}

\begin{table}
\caption{Units for Magnetic Properties}
\label{table}
\setlength{\tabcolsep}{3pt}
\begin{tabular}{|p{25pt}|p{75pt}|p{115pt}|}
\hline
Symbol& 
Quantity& 
Conversion from Gaussian and \par CGS EMU to SI $^{\mathrm{a}}$ \\
\hline
$\Phi $& 
magnetic flux& 
1 Mx $\to  10^{-8}$ Wb $= 10^{-8}$ V$\cdot $s \\
$B$& 
magnetic flux density, \par magnetic induction& 
1 G $\to  10^{-4}$ T $= 10^{-4}$ Wb/m$^{2}$ \\
$H$& 
magnetic field strength& 
1 Oe $\to  10^{3}/(4\pi )$ A/m \\
$m$& 
magnetic moment& 
1 erg/G $=$ 1 emu \par $\to 10^{-3}$ A$\cdot $m$^{2} = 10^{-3}$ J/T \\
$M$& 
magnetization& 
1 erg/(G$\cdot $cm$^{3}) =$ 1 emu/cm$^{3}$ \par $\to 10^{3}$ A/m \\
4$\pi M$& 
magnetization& 
1 G $\to  10^{3}/(4\pi )$ A/m \\
$\sigma $& 
specific magnetization& 
1 erg/(G$\cdot $g) $=$ 1 emu/g $\to $ 1 A$\cdot $m$^{2}$/kg \\
$j$& 
magnetic dipole \par moment& 
1 erg/G $=$ 1 emu \par $\to 4\pi \times  10^{-10}$ Wb$\cdot $m \\
$J$& 
magnetic polarization& 
1 erg/(G$\cdot $cm$^{3}) =$ 1 emu/cm$^{3}$ \par $\to 4\pi \times  10^{-4}$ T \\
$\chi , \kappa $& 
susceptibility& 
1 $\to  4\pi $ \\
$\chi_{\rho }$& 
mass susceptibility& 
1 cm$^{3}$/g $\to  4\pi \times  10^{-3}$ m$^{3}$/kg \\
$\mu $& 
permeability& 
1 $\to  4\pi \times  10^{-7}$ H/m \par $= 4\pi \times  10^{-7}$ Wb/(A$\cdot $m) \\
$\mu_{r}$& 
relative permeability& 
$\mu \to \mu_{r}$ \\
$w, W$& 
energy density& 
1 erg/cm$^{3} \to  10^{-1}$ J/m$^{3}$ \\
$N, D$& 
demagnetizing factor& 
1 $\to  1/(4\pi )$ \\
\hline
\multicolumn{3}{p{251pt}}{Vertical lines are optional in tables. Statements that serve as captions for 
the entire table do not need footnote letters. }\\
\multicolumn{3}{p{251pt}}{$^{\mathrm{a}}$Gaussian units are the same as cg emu for magnetostatics; Mx 
$=$ maxwell, G $=$ gauss, Oe $=$ oersted; Wb $=$ weber, V $=$ volt, s $=$ 
second, T $=$ tesla, m $=$ meter, A $=$ ampere, J $=$ joule, kg $=$ 
kilogram, H $=$ henry.}
\end{tabular}
\label{tab1}
\end{table}

\subsection{Multipart figures}
Figures compiled of more than one sub-figure presented side-by-side, or 
stacked. If a multipart figure is made up of multiple figure
types (one part is lineart, and another is grayscale or color) the figure 
should meet the stricter guidelines.

\subsection{File Formats For Graphics}\label{formats}
Format and save your graphics using a suitable graphics processing program 
that will allow you to create the images as PostScript (PS), Encapsulated 
PostScript (.EPS), Tagged Image File Format (.TIFF), Portable Document 
Format (.PDF), Portable Network Graphics (.PNG), or Metapost (.MPS), sizes them, and adjusts 
the resolution settings. When 
submitting your final paper, your graphics should all be submitted 
individually in one of these formats along with the manuscript.

\subsection{Sizing of Graphics}
Most charts, graphs, and tables are one column wide (3.5 inches/88 
millimeters/21 picas) or page wide (7.16 inches/181 millimeters/43 
picas). The maximum depth a graphic can be is 8.5 inches (216 millimeters/54
picas). When choosing the depth of a graphic, please allow space for a 
caption. Figures can be sized between column and page widths if the author 
chooses, however it is recommended that figures are not sized less than 
column width unless when necessary. 

There is currently one publication with column measurements that do not 
coincide with those listed above. Proceedings of the IEEE has a column 
measurement of 3.25 inches (82.5 millimeters/19.5 picas). 

The final printed size of author photographs is exactly
1 inch wide by 1.25 inches tall (25.4 millimeters$\,\times\,$31.75 millimeters/6 
picas$\,\times\,$7.5 picas). Author photos printed in editorials measure 1.59 inches 
wide by 2 inches tall (40 millimeters$\,\times\,$50 millimeters/9.5 picas$\,\times\,$12 
picas).

\subsection{Resolution }
The proper resolution of your figures will depend on the type of figure it 
is as defined in the ``Types of Figures'' section. Author photographs, 
color, and grayscale figures should be at least 300dpi. Line art, including 
tables should be a minimum of 600dpi.

\subsection{Vector Art}
In order to preserve the figures' integrity across multiple computer 
platforms, we accept files in the following formats: .EPS/.PDF/.PS. All 
fonts must be embedded or text converted to outlines in order to achieve the 
best-quality results.

\subsection{Color Space}
The term color space refers to the entire sum of colors that can be 
represented within the said medium. For our purposes, the three main color 
spaces are Grayscale, RGB (red/green/blue) and CMYK 
(cyan/magenta/yellow/black). RGB is generally used with on-screen graphics, 
whereas CMYK is used for printing purposes.

All color figures should be generated in RGB or CMYK color space. Grayscale 
images should be submitted in Grayscale color space. Line art may be 
provided in grayscale OR bitmap colorspace. Note that ``bitmap colorspace'' 
and ``bitmap file format'' are not the same thing. When bitmap color space 
is selected, .TIF/.TIFF/.PNG are the recommended file formats.

\subsection{Accepted Fonts Within Figures}
When preparing your graphics IEEE suggests that you use of one of the 
following Open Type fonts: Times New Roman, Helvetica, Arial, Cambria, and 
Symbol. If you are supplying EPS, PS, or PDF files all fonts must be 
embedded. Some fonts may only be native to your operating system; without 
the fonts embedded, parts of the graphic may be distorted or missing.

A safe option when finalizing your figures is to strip out the fonts before 
you save the files, creating ``outline'' type. This converts fonts to 
artwork what will appear uniformly on any screen.

\subsection{Using Labels Within Figures}

\subsubsection{Figure Axis labels }
Figure axis labels are often a source of confusion. Use words rather than 
symbols. As an example, write the quantity ``Magnetization,'' or 
``Magnetization M,'' not just ``M.'' Put units in parentheses. Do not label 
axes only with units. As in Fig. 1, for example, write ``Magnetization 
(A/m)'' or ``Magnetization (A$\cdot$m$^{-1}$),'' not just ``A/m.'' Do not label axes with a ratio of quantities and 
units. For example, write ``Temperature (K),'' not ``Temperature/K.'' 

Multipliers can be especially confusing. Write ``Magnetization (kA/m)'' or 
``Magnetization (10$^{3}$ A/m).'' Do not write ``Magnetization 
(A/m)$\,\times\,$1000'' because the reader would not know whether the top 
axis label in Fig. 1 meant 16000 A/m or 0.016 A/m. Figure labels should be 
legible, approximately 8 to 10 point type.

\subsubsection{Subfigure Labels in Multipart Figures and Tables}
Multipart figures should be combined and labeled before final submission. 
Labels should appear centered below each subfigure in 8 point Times New 
Roman font in the format of (a) (b) (c). 

\subsection{File Naming}
Figures (line artwork or photographs) should be named starting with the 
first 5 letters of the author's last name. The next characters in the 
filename should be the number that represents the sequential 
location of this image in your article. For example, in author 
``Anderson's'' paper, the first three figures would be named ander1.tif, 
ander2.tif, and ander3.ps.

Tables should contain only the body of the table (not the caption) and 
should be named similarly to figures, except that `.t' is inserted 
in-between the author's name and the table number. For example, author 
Anderson's first three tables would be named ander.t1.tif, ander.t2.ps, 
ander.t3.eps.

Author photographs should be named using the first five characters of the 
pictured author's last name. For example, four author photographs for a 
paper may be named: oppen.ps, moshc.tif, chen.eps, and duran.pdf.

If two authors or more have the same last name, their first initial(s) can 
be substituted for the fifth, fourth, third$\ldots$ letters of their surname 
until the degree where there is differentiation. For example, two authors 
Michael and Monica Oppenheimer's photos would be named oppmi.tif, and 
oppmo.eps.

\subsection{Referencing a Figure or Table Within Your Paper}
When referencing your figures and tables within your paper, use the 
abbreviation ``Fig.'' even at the beginning of a sentence. Do not abbreviate 
``Table.'' Tables should be numbered with Roman Numerals.

\subsection{Checking Your Figures: The IEEE Graphics Analyzer}
The IEEE Graphics Analyzer enables authors to pre-screen their graphics for 
compliance with IEEE Transactions and Journals standards before submission. 
The online tool, located at
\underline{http://graphicsqc.ieee.org/}, allows authors to 
upload their graphics in order to check that each file is the correct file 
format, resolution, size and colorspace; that no fonts are missing or 
corrupt; that figures are not compiled in layers or have transparency, and 
that they are named according to the IEEE Transactions and Journals naming 
convention. At the end of this automated process, authors are provided with 
a detailed report on each graphic within the web applet, as well as by 
email.

For more information on using the Graphics Analyzer or any other graphics 
related topic, contact the IEEE Graphics Help Desk by e-mail at 
graphics@ieee.org.

\subsection{Submitting Your Graphics}
Because IEEE will do the final formatting of your paper,
you do not need to position figures and tables at the top and bottom of each 
column. In fact, all figures, figure captions, and tables can be placed at 
the end of your paper. In addition to, or even in lieu of submitting figures 
within your final manuscript, figures should be submitted individually, 
separate from the manuscript in one of the file formats listed above in 
Section \ref{formats}. Place figure captions below the figures; place table titles 
above the tables. Please do not include captions as part of the figures, or 
put them in ``text boxes'' linked to the figures. Also, do not place borders 
around the outside of your figures.

\subsection{Color Processing/Printing in IEEE Journals}
All IEEE Transactions, Journals, and Letters allow an author to publish 
color figures on IEEE Xplore\textregistered\ at no charge, and automatically 
convert them to grayscale for print versions. In most journals, figures and 
tables may alternatively be printed in color if an author chooses to do so. 
Please note that this service comes at an extra expense to the author. If 
you intend to have print color graphics, include a note with your final 
paper indicating which figures or tables you would like to be handled that 
way, and stating that you are willing to pay the additional fee.

\section{Conclusion}
A conclusion section is not required. Although a conclusion may review the 
main points of the paper, do not replicate the abstract as the conclusion. A 
conclusion might elaborate on the importance of the work or suggest 
applications and extensions. 

\appendices

Appendixes, if needed, appear before the acknowledgment.

\section*{Acknowledgment}

The preferred spelling of the word ``acknowledgment'' in American English is 
without an ``e'' after the ``g.'' Use the singular heading even if you have 
many acknowledgments. Avoid expressions such as ``One of us (S.B.A.) would 
like to thank $\ldots$ .'' Instead, write ``F. A. Author thanks $\ldots$ .'' In most 
cases, sponsor and financial support acknowledgments are placed in the 
unnumbered footnote on the first page, not here.

\section*{References and Footnotes}

\subsection{References}
References need not be cited in text. When they are, they appear on the 
line, in square brackets, inside the punctuation. Multiple references are 
each numbered with separate brackets. When citing a section in a book, 
please give the relevant page numbers. In text, refer simply to the 
reference number. Do not use ``Ref.'' or ``reference'' except at the 
beginning of a sentence: ``Reference \cite{b3} shows $\ldots$ .'' Please do not use 
automatic endnotes in \emph{Word}, rather, type the reference list at the end of the 
paper using the ``References'' style.

Reference numbers are set flush left and form a column of their own, hanging 
out beyond the body of the reference. The reference numbers are on the line, 
enclosed in square brackets. In all references, the given name of the author 
or editor is abbreviated to the initial only and precedes the last name. Use 
them all; use \emph{et al.} only if names are not given. Use commas around Jr., 
Sr., and III in names. Abbreviate conference titles. When citing IEEE 
transactions, provide the issue number, page range, volume number, year, 
and/or month if available. When referencing a patent, provide the day and 
the month of issue, or application. References may not include all 
information; please obtain and include relevant information. Do not combine 
references. There must be only one reference with each number. If there is a 
URL included with the print reference, it can be included at the end of the 
reference. 

Other than books, capitalize only the first word in a paper title, except 
for proper nouns and element symbols. For papers published in translation 
journals, please give the English citation first, followed by the original 
foreign-language citation See the end of this document for formats and 
examples of common references. For a complete discussion of references and 
their formats, see the IEEE style manual at
\underline{http://www.ieee.org/authortools}.

\subsection{Footnotes}
Number footnotes separately in superscript numbers.\footnote{It is recommended that footnotes be avoided (except for 
the unnumbered footnote with the receipt date on the first page). Instead, 
try to integrate the footnote information into the text.} Place the actual 
footnote at the bottom of the column in which it is cited; do not put 
footnotes in the reference list (endnotes). Use letters for table footnotes 
(see Table \ref{table}).

\section{Submitting Your Paper for Review}

\subsection{Final Stage}
When you submit your final version (after your paper has been accepted), 
print it in two-column format, including figures and tables. You must also 
send your final manuscript on a disk, via e-mail, or through a Web 
manuscript submission system as directed by the society contact. You may use 
\emph{Zip} for large files, or compress files using \emph{Compress, Pkzip, Stuffit,} or \emph{Gzip.} 

Also, send a sheet of paper or PDF with complete contact information for all 
authors. Include full mailing addresses, telephone numbers, fax numbers, and 
e-mail addresses. This information will be used to send each author a 
complimentary copy of the journal in which the paper appears. In addition, 
designate one author as the ``corresponding author.'' This is the author to 
whom proofs of the paper will be sent. Proofs are sent to the corresponding 
author only.

\subsection{Review Stage Using ScholarOne\textregistered\ Manuscripts}
Contributions to the Transactions, Journals, and Letters may be submitted 
electronically on IEEE's on-line manuscript submission and peer-review 
system, ScholarOne\textregistered\ Manuscripts. You can get a listing of the 
publications that participate in ScholarOne at 
\underline{http://www.ieee.org/publications\_standards/publications/}\discretionary{}{}{}\underline{authors/authors\_submission.html}.
First check if you have an existing account. If there is none, please create 
a new account. After logging in, go to your Author Center and click ``Submit 
First Draft of a New Manuscript.'' 

Along with other information, you will be asked to select the subject from a 
pull-down list. Depending on the journal, there are various steps to the 
submission process; you must complete all steps for a complete submission. 
At the end of each step you must click ``Save and Continue''; just uploading 
the paper is not sufficient. After the last step, you should see a 
confirmation that the submission is complete. You should also receive an 
e-mail confirmation. For inquiries regarding the submission of your paper on 
ScholarOne Manuscripts, please contact oprs-support@ieee.org or call +1 732 
465 5861.

ScholarOne Manuscripts will accept files for review in various formats. 
Please check the guidelines of the specific journal for which you plan to 
submit.

You will be asked to file an electronic copyright form immediately upon 
completing the submission process (authors are responsible for obtaining any 
security clearances). Failure to submit the electronic copyright could 
result in publishing delays later. You will also have the opportunity to 
designate your article as ``open access'' if you agree to pay the IEEE open 
access fee. 

\subsection{Final Stage Using ScholarOne Manuscripts}
Upon acceptance, you will receive an email with specific instructions 
regarding the submission of your final files. To avoid any delays in 
publication, please be sure to follow these instructions. Most journals 
require that final submissions be uploaded through ScholarOne Manuscripts, 
although some may still accept final submissions via email. Final 
submissions should include source files of your accepted manuscript, high 
quality graphic files, and a formatted pdf file. If you have any questions 
regarding the final submission process, please contact the administrative 
contact for the journal. 

In addition to this, upload a file with complete contact information for all 
authors. Include full mailing addresses, telephone numbers, fax numbers, and 
e-mail addresses. Designate the author who submitted the manuscript on 
ScholarOne Manuscripts as the ``corresponding author.'' This is the only 
author to whom proofs of the paper will be sent. 

\subsection{Copyright Form}
Authors must submit an electronic IEEE Copyright Form (eCF) upon submitting 
their final manuscript files. You can access the eCF system through your 
manuscript submission system or through the Author Gateway. You are 
responsible for obtaining any necessary approvals and/or security 
clearances. For additional information on intellectual property rights, 
visit the IEEE Intellectual Property Rights department web page at 
\underline{http://www.ieee.org/publications\_standards/publications/rights/}\discretionary{}{}{}\underline{index.html}. 

\section{IEEE Publishing Policy}
The general IEEE policy requires that authors should only submit original 
work that has neither appeared elsewhere for publication, nor is under 
review for another refereed publication. The submitting author must disclose 
all prior publication(s) and current submissions when submitting a 
manuscript. Do not publish ``preliminary'' data or results. The submitting 
author is responsible for obtaining agreement of all coauthors and any 
consent required from employers or sponsors before submitting an article. 
The IEEE Transactions and Journals Department strongly discourages courtesy 
authorship; it is the obligation of the authors to cite only relevant prior 
work.

The IEEE Transactions and Journals Department does not publish conference 
records or proceedings, but can publish articles related to conferences that 
have undergone rigorous peer review. Minimally, two reviews are required for 
every article submitted for peer review.

\section{Publication Principles}
The two types of contents of that are published are; 1) peer-reviewed and 2) 
archival. The Transactions and Journals Department publishes scholarly 
articles of archival value as well as tutorial expositions and critical 
reviews of classical subjects and topics of current interest. 

Authors should consider the following points:

\begin{enumerate}
\item Technical papers submitted for publication must advance the state of knowledge and must cite relevant prior work. 
\item The length of a submitted paper should be commensurate with the importance, or appropriate to the complexity, of the work. For example, an obvious extension of previously published work might not be appropriate for publication or might be adequately treated in just a few pages.
\item Authors must convince both peer reviewers and the editors of the scientific and technical merit of a paper; the standards of proof are higher when extraordinary or unexpected results are reported. 
\item Because replication is required for scientific progress, papers submitted for publication must provide sufficient information to allow readers to perform similar experiments or calculations and 
use the reported results. Although not everything need be disclosed, a paper 
must contain new, useable, and fully described information. For example, a 
specimen's chemical composition need not be reported if the main purpose of 
a paper is to introduce a new measurement technique. Authors should expect 
to be challenged by reviewers if the results are not supported by adequate 
data and critical details.
\item Papers that describe ongoing work or announce the latest technical achievement, which are suitable for presentation at a professional conference, may not be appropriate for publication.
\end{enumerate}


\bibliographystyle{unsrt}
\bibliography{references}


\begin{IEEEbiography}[{\includegraphics[width=1in,height=1.25in,clip,keepaspectratio]{a1.png}}]{First A. Author} (M'76--SM'81--F'87) and all authors may include 
biographies. Biographies are often not included in conference-related
papers. This author became a Member (M) of IEEE in 1976, a Senior
Member (SM) in 1981, and a Fellow (F) in 1987. The first paragraph may
contain a place and/or date of birth (list place, then date). Next,
the author's educational background is listed. The degrees should be
listed with type of degree in what field, which institution, city,
state, and country, and year the degree was earned. The author's major
field of study should be lower-cased. 

The second paragraph uses the pronoun of the person (he or she) and not the 
author's last name. It lists military and work experience, including summer 
and fellowship jobs. Job titles are capitalized. The current job must have a 
location; previous positions may be listed 
without one. Information concerning previous publications may be included. 
Try not to list more than three books or published articles. The format for 
listing publishers of a book within the biography is: title of book 
(publisher name, year) similar to a reference. Current and previous research 
interests end the paragraph. The third paragraph begins with the author's 
title and last name (e.g., Dr.\ Smith, Prof.\ Jones, Mr.\ Kajor, Ms.\ Hunter). 
List any memberships in professional societies other than the IEEE. Finally, 
list any awards and work for IEEE committees and publications. If a 
photograph is provided, it should be of good quality, and 
professional-looking. Following are two examples of an author's biography.
\end{IEEEbiography}

\begin{IEEEbiography}[{\includegraphics[width=1in,height=1.25in,clip,keepaspectratio]{a2.png}}]{Second B. Author} was born in Greenwich Village, New York, NY, USA in 
1977. He received the B.S. and M.S. degrees in aerospace engineering from 
the University of Virginia, Charlottesville, in 2001 and the Ph.D. degree in 
mechanical engineering from Drexel University, Philadelphia, PA, in 2008.

From 2001 to 2004, he was a Research Assistant with the Princeton Plasma 
Physics Laboratory. Since 2009, he has been an Assistant Professor with the 
Mechanical Engineering Department, Texas A{\&}M University, College Station. 
He is the author of three books, more than 150 articles, and more than 70 
inventions. His research interests include high-pressure and high-density 
nonthermal plasma discharge processes and applications, microscale plasma 
discharges, discharges in liquids, spectroscopic diagnostics, plasma 
propulsion, and innovation plasma applications. He is an Associate Editor of 
the journal \emph{Earth, Moon, Planets}, and holds two patents. 

Dr. Author was a recipient of the International Association of Geomagnetism 
and Aeronomy Young Scientist Award for Excellence in 2008, and the IEEE 
Electromagnetic Compatibility Society Best Symposium Paper Award in 2011. 
\end{IEEEbiography}

\begin{IEEEbiography}[{\includegraphics[width=1in,height=1.25in,clip,keepaspectratio]{a3.png}}]{Third C. Author, Jr.} (M'87) received the B.S. degree in mechanical 
engineering from National Chung Cheng University, Chiayi, Taiwan, in 2004 
and the M.S. degree in mechanical engineering from National Tsing Hua 
University, Hsinchu, Taiwan, in 2006. He is currently pursuing the Ph.D. 
degree in mechanical engineering at Texas A{\&}M University, College 
Station, TX, USA.

From 2008 to 2009, he was a Research Assistant with the Institute of 
Physics, Academia Sinica, Tapei, Taiwan. His research interest includes the 
development of surface processing and biological/medical treatment 
techniques using nonthermal atmospheric pressure plasmas, fundamental study 
of plasma sources, and fabrication of micro- or nanostructured surfaces. 

Mr. Author's awards and honors include the Frew Fellowship (Australian 
Academy of Science), the I. I. Rabi Prize (APS), the European Frequency and 
Time Forum Award, the Carl Zeiss Research Award, the William F. Meggers 
Award and the Adolph Lomb Medal (OSA).
\end{IEEEbiography}

\end{document}
