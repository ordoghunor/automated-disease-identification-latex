\section{Related work}

In the field of dental diagnostics, researchers have embarked on a journey to improve disease detection and diagnosis through a variety of innovative technological approaches.

Michael G. Endres and Florian Hillen delved into the field of deep learning algorithms for periapical disease detection in dental radiographs\cite{endres2020development}. Their research showed that the deep learning algorithm outperformed a significant number of oral and maxillofacial surgeons in detecting periapical lucencies. This demonstrates the potential of deep learning techniques to augment the skills of dentists, especially in challenging diagnostic tasks.

Jae-Hong Lee and Do-Hyung Kim in 2020 explored the application of deep convolutional neural networks (CNNs) for the detection and diagnosis of dental caries using periapical radiographs\cite{lee2018detection}. Their study showed impressive diagnostic accuracies and area under the curve (AUC) values, indicating the robustness of deep learning models in caries detection. This highlights the potential of CNNs as a valuable tool to assist dental professionals in early disease detection and diagnosis.

Hu Chen and Hong Li investigated the use of deep CNNs with region proposal techniques for detecting dental disease on periapical radiographs\cite{chen2021dental}. Through their research, they provided valuable insights into the optimization of CNN architectures for disease detection in dental imaging. This research contributes to the growing body of knowledge on the use of advanced image analysis techniques to improve diagnostic accuracy in dental radiology.

AL-Ghamdi and Mahmoud Ragab in 2022 proposed a convolutional neural network architecture for multitask classification of dental disease on panoramic radiographs\cite{al2022detection}. Their CNN model exhibited an impressive accuracy rate of over 96\%, highlighting its potential as a robust tool for dental disease detection. This highlights the importance of using advanced machine learning techniques to automate disease classification tasks in dental diagnostics.

Whisnu U. Setiabudi and Endang Sugiharti's work introduced an expert system for diagnosing dental diseases using the Certainty Factor method\cite{setiabudi2017expert}. Their Android-based application achieved a high accuracy rate of 95\% in diagnosing dental diseases based on patient symptoms. This highlights the effectiveness of expert systems in providing accurate and reliable diagnoses, thereby facilitating timely intervention and treatment planning.

Andi M. A. K. Parewe and Wayadan F. Mahmudy proposed a hybrid approach combining fuzzy logic and evolutionary strategies for dental disease detection\cite{parewe2018dental}. Their method showed improved accuracy compared to traditional fuzzy logic systems. By integrating evolutionary algorithms into the diagnostic process, this research opens new avenues for improving the efficiency and accuracy of dental disease detection systems.